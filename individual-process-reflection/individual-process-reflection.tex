\documentclass[a4paper,12pt]{article}


\usepackage[left=1cm,right=1cm,top=2cm,bottom=2cm]{geometry}
\usepackage{graphicx}
\usepackage{parskip}

\author{Taharka Okai}
\title{Individual Process Reflection}

\begin{document}

\begin{titlepage}
  \begin{center}
    \vspace*{\fill}

    \includegraphics[width=0.5\textwidth]{
      graphics/university-of-bristol-logo-png-transparent}
    \\[1cm]

    {\bfseries\Huge
    Individual Process Reflection}
    \\[1cm]

    {\Huge
    The Complementary Relationship between Value Proposition
    and the Business Model in Innovation}
    \\[1cm]

    {\Large
    INNOVM0015 -- Innovation, Entrepreneurship and Enterprise}
    \\[1cm]

    {\Large
    Student : Taharka Okai \\[.125cm]
    Date    : 23/11/2022   \\[.125cm]
    Word Count    :   \\[.125cm]
    }

    \vspace*{\fill}
  \end{center}
\end{titlepage}

\section{Introduction}
\label{sec:Introduction}

Product ideas that businesses produce during ideation feature precedents
that attract recipients in the form of consumers. These precedents can
be summarised in the form of a Value Proposition. Once the reasons for
the product's success are established, then the innovation process can
continue to focus on delivering that project. However, innovation is an
iterative process, and

\section{Job to be Done}
\label{sec:Job to be Done}

One Value Proposition Technique is the `Job to be Done' framework. This
framework defines a method that helps to ensure that the product
idea being formulated meets some unaddressed issue or improves on an
existing solution.

% About the jobs to be done framework

% Usage


\section{Business Model Canvas}
\label{sec:Business Model Canvas}

% About the business model canvas

% Usage

\section{Cause and Effect}
\label{sec:Cause and Effect}

While applying the Business Model Canvas, our team became aware of an
issue with our Value Proposition.


\end{document}